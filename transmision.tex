La transmisión de votos para su posterior escrutinio también tiene su complejidad ya que la comunicación de estos es sensible a vulnerabilidades y pueden ocurrir ataques como los enunciados a continuacion:

\subsubsection{Software o hardware malicioso que puede cambiar la intención del voto, agregar nuevos votos o eliminar votos}	
Detalle del ataque multivoto en CABA\footnote{\url{https://docs.google.com/document/d/1aH6kvoLR8O1qWOpEz89FAB2xFcBNB-QqHgZpXxg0vGE/preview}}.\\

El sistema de almacenamiento de votos en boletas electronicas en CABA consiste en la grabación del voto en un tag RFID, que luego se utiliza para realizar el conteo de votos de forma electronica. En este caso la vulnerabilidad proviene de un software mal diseñado y/o implementado que no contiene las debidas consideraciones al momento de leer e interpretar los datos de una boleta.\\

Mas particularmente, puede observarse en una porción del código del sistema de voto electronico lo siguiente:

\begin{lstlisting}
class Selection(object):
 ...

    def from_string(TAGcontent):
        datatag = parse(TAGcontent)

 ...
 
      candidates = []
        for e in datatag.vote:
            party_code = e["party"]
            category_code = e["category"]
            candidate = CandidateClass.get(category_code,
                                           party_code)
            candidates.append(candidate)

\end{lstlisting}

Mas allá de una falla de verificacion en la funcion \texttt{parse} mencionada en la bibliografia, lo mas grave es que nunca se chequea ninguna de las siguientes cosas:
\begin{itemize}
	\item Unicidad de voto por boleta
	\item Cantidad de votos por candidato sea a lo sumo 1
\end{itemize} 

Continuando con la descripción de la vulnerabilidad, en otra sección vital del código, el recuento de votos, se encuentra el siguiente código:

\begin{lstlisting}
class Count(object):

 ...
   def add_selection(self, selection, RFIDserial=None):
        if not RFIDserial or not self.serial_exists(RFIDserial):
            for candidate in selection.candidates:
                self.results[candidate.party_code, candidate.category_code] += 1
            if RFIDserial:
                self._RFIDserials.append(RFIDserial)
  ...
        else:
            raise RepeatedSerial()
\end{lstlisting}

Aquí puede verse una validación sobre la identificación única de cada boleta, pero nunca se realiza la validación pertinente a cada boleta en si misma acerca de los 2 puntos mencionados anteriormente.\\

Esta vulnerabilidad, permitia al atacante, escribir sobre una boleta electronica mediante un smartphone, varios votos a un mismo candidato y esto pasaría desapercibido por ambos módulos del sistema mostrados arriba, \textbf{constituyendo una fuerte vulnerabilidad}.

A continuación se mencionan posibles soluciones a este problema:

\begin{itemize}
	\item Verificar que la escritura de los votos hayan provenido de una maquina real de vot.ar, por ejemplo usando claves privadas.
	\item Verificar al momento de la lectura y recuento de la boleta las cuestiones mencionadas
	\item Como método de deteccion se pueden comparar la cantidad de boletas y la cantidad de votos registrados por máquina.
\end{itemize}


\subsubsection{Ataques de denegación de servicio por botnets causando que no se contabilicen votos}

El mecanismo de transmisión de datos utilizado en el sistema Vot.Ar consiste básicamente en una conexión entre cada terminal de votación y un servidor central que recibe los votos a medida que van siendo leídos las boletas con RFID. \\

Si alguien con acceso a la red de alguna máquina de votacion quisiera atacar la disponibilidad del servidor para, por ejemplo, retrasar los resultados, podría simplemente obtener la IP o nombre del servidor central y mediante alguna red de botnets obtenida por otro medio, realizar ataques de denegación de servicio y posiblemente explotar vulnerabilidades en el servidor. Por ejemplo podría realizarse un \texttt{SYN-Flood attack}, alterando la disponibilidad del servidor.
	
\subsubsection{El envío de boletas para el voto por internet puede pasar por muchos y diferentes servidores, logrando que el sistema no tenga la certeza de que la boleta recibida por el elector, es la boleta emitida para ser usada durante la votación. Además los emails pueden ser interceptados, leídos y modificados en el tránsito violando todos los requerimientos para realizar una votación}

	\begin{itemize}
		\item En Estados Unidos los estándares de transimisión de votos son muy variables. Cada estado elije las máquinas y los métodos para usar y no se rigen por ningún mínimo de seguridad común. Hay un requerimiento estádistico de un error no mayor a 1 en 1 millón, que aparentemente no se cumple\footnote{Douglas W. Jones, \textit{Problems with Voting Systems and the Applicable Standards}, Testimony before the U.S. House of Representatives' Committee on Science, \url{http://homepage.cs.uiowa.edu/~jones/voting/congress.html}}.
	\end{itemize}

\subsubsection{Es posible que un empleado del gobierno encargado de información sensible tuviera interés en alterar los resultados por alguna vía. Mala praxis informática, de alguna forma}
	
	\begin{itemize}
		\item En CABA la empresa encargada de implementar el voto electronico  puso en manos de sus técnicos contratados la seguridad de la transmisión de datos del escrutinio provisorio. Además, lo hizo implementando un sistema de usuarios y contraseñas fácilmente deducible. Y para empeorar aún más la situación, una vez que alguien (un técnico o alguien que tuviera acceso a esta información) lograba hacerse de un certificado y una clave para la transmisión de datos, podía fácilmente descargar los de todos los centros de votación\footnote{Francisco Amato, Iván A. Barrera Oro, Enrique Chaparro, Sergio Demian Lerner, Alfredo Ortega, Juliano Rizzo, Fernando Russ, Javier Smaldone, Nicolas Waisman, \textit{El día que el sistema de voto electrónico Vot.Ar fue vulnerado}, \url{https://blog.smaldone.com.ar/2016/05/03/el-dia-que-el-sistema-de-voto-electronico-vot-ar-fue-vulnerado/}}.
	\end{itemize}

