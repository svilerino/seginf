\textit{Movilidad y accesibilidad:} “Un votante no debería estar restringido a votar en un solo lugar”. Lo relacionamos con garantizar la disponibilidad. En la realidad y a nivel nacional, se lo designa a un solo lugar, pero las máquinas no hacen un chequeo de esto. Por lo tanto, se estaría permitiendo este fundamento. Según el Gobierno de la Ciudad de Buenos Aires, el padrón electoral no se ha digitalizado. (FAQ 11, GCBA Boleta electrónica). Tampoco se limitan los votantes una máquina únicamente, ante un caso de falla o inutilización de la misma, debería ser fácilmente reemplazables. El gobierno de la ciudad afirma que esto es posible (FAQ 3, GCBA Boleta electrónica).
En cuanto a los votantes con discapacidades, el sistema debe facilitar y hacer posible el uso para los mismos. Según el GCBA, se puede activar un modo de alto contraste, o sino un teclado físico especial en braille con audífonos para mantener la privacidad del votante y escuchar instrucciones. En cuanto a seguridad, comprendemos que debe tener en cuenta dos cosas. Garantizar el acceso a todos los votantes y además, no generar vulnerabilidades habilitando estos modos. Utilizando una máquina que garantice todo lo expresado anteriormente, ayuda a garantizar el voto universal y secreto.

\textit{Conveniente:} el sistema debería permitir al votante emitir el voto de forma rápida, en una sesión, y no requerir habilidades especiales, ni intimidar al votante. (Ley 4894 C.A.B.A., artículo 24, inciso a). Previo utilización del sistema electrónico, se pusieron puntos de capacitación o consulta sobre la utilización del mismo. Lo que podría ser cuestionable es el tiempo previo al mismo para garantizar este fundamento. La Defensoría del Pueblo, lanzó unos pocos meses antes una campaña de capacitación, abril del 2015 (http://www.defensoria.org.ar/biblioteca/, informe anual 2015). Así mismo, se hicieron accesibles las supuestas máquinas a utilizar, lo que facilita un primer vistazo a posibles vulnerabilidades de las mismas por parte de algún posible atacante. Como por ejemplo, si tienen expuestos puertos o entrada y salida de medios. Por otra parte, las indicaciones visibles, de cómo utilizar la máquina tienen que ser fáciles de interpretar. (Ley 4894 C.A.B.A., artículo 24, inciso f)

\textit{Unicidad:} un votante puede votar una única vez. El control del mismo se realiza al igual que la elegibilidad. El identificador de una boleta no está asociado al votante una vez insertado la boleta impresa en la urna. El control lo hacen las autoridades de mesa, que deben brindar una única boleta al votante.

\textit{Autenticidad del votante, elegibilidad:} se tiene que asegurar que el votante se identifique. Se realiza de forma personal. Una vez identificado, se le entrega una única boleta. Si no se identifica, no debería votar. Nuevamente, el control es humano.

\textit{El voto secreto del votante:} como la ley establece que el voto es secreto, el sistema debe de garantizar esto. El sistema utilizado en C.A.B.A. asegura que garantiza el mismo ya que las máquinas no poseen memoria que grabe información del voto y la boleta al doblarse correctamente, posee una protección contra lecturas externas. Solo abriéndose y apoyándola en una máquina lectora, se puede conocer el voto. A su vez, una vez introducida la boleta en la urna, se pierde la relación de la boleta con el votante. (FAQ GCBA Boleta electrónica). La boleta no almacena ninguna información propia del votante.
En un informe del Poder Ciudadano, sobre las elecciones en Salta (mismo sistema usado en las elecciones de la Ciudad de Buenos Aires), se destacó como una gran debilidad la pérdida del voto secreto. Se reportó que, personas con discapacidad, requirieron de ayuda para poder emitir su voto. La asistencia fue sin una distancia prudente. La accesibilidad del sufragio debe garantizarse con el ejercicio individual del voto.

\textit{Integridad:} este fundamento también aparece en la ley (Ley 4894 C.A.B.A., artículo 24, inciso g) del voto en CABA. Las máquinas no tienen almacenamiento ni guardan información de la votación, por lo que la integridad de las mismas se corresponde a la integridad del software que ejecutan. El mismo viene en un DVD, sellado con un hash disponible para verificar la integridad del mismo. El hash del contenido era con md5 luego cambiado a SHA512, pero no se utiliza. El contenido hasheado se almacenaba en un archivo que puede ser generado por cualquier persona. No está firmado digitalmente, dando cuenta de un grave error de apreciación de seguridad. (informe Smaldone). Así mismo, el hash del DVD se almacena en el propio DVD, otro gravísimo error, ya que alcanza con que alguien modificara el DVD, calculara el nuevo hash y lo pusiera en el mismo lugar. Las autoridades de mesa no podrían verificar la integridad del mismo con respecto del DVD oficial.
En el caso de la boleta electrónica, posee una memoria que permite la posibilidad de una única escritura, bloqueando otras posibles cuando se intentara grabar. La responsabilidad es de quien graba la información por primera vez. Por lo que, si bien las máquinas utilizadas activaban dicho bloqueo, si un usuario pudiese violar la máquina, podría hacer que no se active la escritura única, pudiendo modificar información ya almacenada en la BUE (boleta única electrónica), según el informe de Smaldone donde se hace referencia al manual del chip RFID utilizado. La importancia de la escritura única es garantizar la integridad de un voto ya emitido, bloqueando cualquier intento de cambiar el contenido del voto.
Como se mostró, el equipo utilizado permite un fácil acceso a los puertos USB, lectora de DVD, etc. Lo que uno podría insertar un software para modificar el normal funcionamiento del sistema y corromper el la información que se muestra con la que se imprime en la boleta.
En la auditoría de la FCEN, se destaca que todos los procesos que se ejecutan en la máquina, corren con nivel de root. Por lo que, si un usuario con acceso a la máquina logra aprovechar alguna vulnerabilidad, puede obtener control total del sistema y modificar su comportamiento oficial. Por ejemplo, imprimir cierto valor en la boleta y mostrar otro por pantalla. Al tener ciertos puertos y medios fácilmente accesibles, esto es posible, por ejemplo, explotando vulnerabilidades de los puertos USB.

\textit{Privacidad:} Nadie debería poder determinar el voto de un votante. No existe una relación estricta entre la boleta y el votante, la boleta la elige el votante al azar. Una vez insertada en la urna, se pierda la relación votante/boleta. Además, la máquina no almacena votos por lo que la privacidad, se podría ver amenazada únicamente si lee la boleta a distancia cuando la posee el votante.Según las especificaciones del fabricante de los chips RFID, es posible. (https://blog.smaldone.com.ar/2016/01/08/sobre-el-chip-rfid-de-la-boleta-unica-electronica/). Aunque el GCBA, afirma que se utiliza un film metálico para evitar la lectura de la BUE si está bien doblada.

\textit{Testeo y Certificación:} las máquinas utilizadas no son certificadas. Gran diferencia con Estados Unidos, donde la National Institute of Standards and Technology determina que cada Estado fije distintos grados mínimos de testeo y certificaciones. A su vez, desarrollaron una guía para implementar distintas políticas de seguridad. A nivel local, no hubo una política fuerte por parte del Estado o de la Justicia para garantizar que se cumplan con las políticas de seguridad definidas en la Ley. Hubo resistencia por parte de la empresa MSA para brindar información y prestarse para la auditoría previa, durante y posterior a la elección (Informes en distintos medios y Smaldone).
