\documentclass[a4,11pt]{article}

\parindent=10pt
\parskip=6pt
\usepackage[paper=a4paper, left=2cm, right=2cm, bottom=2.5cm,top=2.5cm]{geometry}

% Paquetes de nacionalización. No olvidar para poder poner tildes!
\usepackage[spanish]{babel}
\usepackage[utf8]{inputenc}
\usepackage{listingsutf8}

% Paquetes para pseudo
\usepackage{listings} % este es para codigo posta
\usepackage[usenames,dvipsnames,svgnames,table]{xcolor}

% Caratula (Recordar logo_uba.jpg y logo_dc.jpg)
\usepackage{caratula}

% Bibliografia
% \usepackage{biblatex}
% \addbibresource{biblio.bib}

% Color de links
\usepackage{hyperref}
\hypersetup{
    colorlinks,
    citecolor=black,
    filecolor=black,
    linkcolor=black,
    urlcolor=black
}

\begin{document}


\materia{Seguridad de la Información}
\submateria{Primer Cuatrimestre 2016}
\titulo{Investigación en seguridad en procesos\\ electorales}
\subtitulo{TP final}
\integrante{Brian Litwak}{}{brian.litwak@gmail.com}
\integrante{Gino Scarpino}{}{gino.scarpino@gmail.com}
\integrante{Silvio Vileriño}{}{svilerino@gmail.com}
\integrante{Valeria Tiffenberg}{193/10}{valetiff@gmail.com}


\maketitle
\pagebreak

\tableofcontents

\pagebreak

\section{Introducción}

En este trabajo se quiere presentar un panorama de la evolución de los sistemas electorales informáticos, tomando los ejemplos de varios países y haciendo hincapié en sus aciertos y errores de seguridad.

Dentro de los sistemas electrónicos de votación, reconocemos dos categorías: el voto electrónico presencial, ejemplificado por el que se dio en las últimas elecciones en la Ciudad de Buenos Aires, y el voto por internet, como lo ha tenido Suiza desde fines de la década del 90\footnote{Jan Gerlach and Urs Gasser,\textit{Three Case Studies from Switzerland: E-Voting}, \url{http://cyber.law.harvard.edu/sites/cyber.law.harvard.edu/files/Gerlach-Gasser_SwissCases_Evoting.pdf}}.

La historia de los sistemas elctrónicos de votación empieza mucho antes que en la Argentina. Sin ir más lejos, Brasil tiene su sistema en funcionamiento desde el año 96\footnote{Corte Electoral Superior de Brasil, \url{http://english.tse.jus.br/electronic-voting/the-electronic-ballot-box-history}}. Otros países con sistemas activos son India, Perú, Colombia, Estados Unidos, Francia y Japón, y ha habido algunos países donde se usó por varios años y decidió discontinuarse, y otros donde se han hecho pilotos que finalizaron en la decisión de no aplicar el sistema\footnote{National Democratic Institute, \url{https://www.ndi.org/e-voting-guide/electronic-voting-and-counting-around-the-world}}.

En este informe todos estos casos resultan de interés, con foco particular en el uso argentino. En el desarrollo del informe se irán mencionando otras experiencias para completar el panorama global relacionado con el voto electrónico.

\section{Temas a desarrollar}

Intentaremos hacer hincapié separadamente en los problemas posibles dentro de la seguridad informática: confidencialidad, integridad y disponibilidad, para tratar de abarcar toda la superficie de vulnerabilidades. Para ordenar el trabajo, se van a usar los grupos sugeridos en el enunciado:

\begin{itemize}
\item Estándares de almacenamiento de votos (Puede afectar disponibilidad e integridad)
\item Seguridad física de las maquinas de votación (integridad)
\item Transmisión de votos para su posterior escrutinio (afecta confidencialidad e integridad)
\item Privacidad del voto (afecta confidencialidad e integridad)
\item Auditabilidad del sistema  (afecta disponibilidad y confidencialidad)
\item Comprobantes
\end{itemize}

Los tres ángulos de la seguridad informática son fundamentales a la confianza en las elecciones electrónicas, y la confianza es, discutiblemente, el pilar sobre el que las elecciones democtráticas suceden. Así como en muchos países no es obligatorio ir a votar, son aquellos que sí confían en que el resultado generará alguna diferencia y tiene importancia, quiénes se acercan a las urnas, digitales o no.

La confidencialidad está establecida por ley en aquellos países que hemos investigado: el voto es secreto para proteger a la persona de ataques o influencias, y que vote a quién realmente desea votar. Las interfaces electrónicas de votación deben garantizar que al momento de votar, nadie más que el votante sabe el voto, y al mismo tiempo, que el almacenamiento de los votos y de la lista de votantes no se pueden unir, a posteriori, para asociar un voto a una persona.

La integridad es, en este contexto, lo más importante para la confianza. Es necesario que el voto que la persona emitió sea contado como tal, ni por otro votante, ni mútiples veces, ni ignorado. El dato como se emitió, debe llegar a destino.

Por último, la disponibilidad tiene varios factores: por un lado es importante que hardware y software de votación estén en funcionamiento correcto al momento de realizarse las elecciones, para todos los votantes; y por otro, es necesario que los votos puedan ser contados para el recuento original, pero incluso que estén disponibles para nuevos recuentos si la justicia lo requiriera, por ejemplo.

\pagebreak
\section{Estándares de almacenamiento de votos}

\pagebreak
\section{Seguridad física de las maquinas de votación}

\pagebreak
\section{Problemas en la transmisión de votos y consecuencias sobre su posterior escrutinio}
La transmisión de votos para su posterior escrutinio también tiene su complejidad ya que la comunicación de estos es sensible a vulnerabilidades y pueden ocurrir ataques.\\
Los casos investigados respecto a los paises Israel e India tienen como particularidad que no transmiten los votos, sino que trasladan las máquinas fisicamente a un lugar y realizan el recuento a mano. \\

\subsection{Voto electronico en CABA (Vot.Ar)}
A continuación haremos referencia al sistema de votación electrónica Vot.Ar utilizado en Argentina en 2015.

\subsubsection{Software o hardware malicioso que puede cambiar la intención del voto, agregar nuevos votos o eliminar votos}
Detalle del ataque multivoto en CABA\footnote{\url{https://docs.google.com/document/d/1aH6kvoLR8O1qWOpEz89FAB2xFcBNB-QqHgZpXxg0vGE/preview}}.

El sistema de almacenamiento de votos en boletas electrónicas en CABA consiste en la grabación del voto en un tag RFID, que luego se utiliza para realizar el conteo de votos de forma electrónica. En este caso la vulnerabilidad proviene de un software mal diseñado y/o implementado que no contiene las debidas consideraciones al momento de leer e interpretar los datos de una boleta.

Más particularmente, puede observarse en una porción del código del sistema de voto electrónico lo siguiente:

\begin{lstlisting}
class Selection(object):
 ...

    def from_string(TAGcontent):
        datatag = parse(TAGcontent)

 ...

      candidates = []
        for e in datatag.vote:
            party_code = e["party"]
            category_code = e["category"]
            candidate = CandidateClass.get(category_code,
                                           party_code)
            candidates.append(candidate)

\end{lstlisting}

Mas allá de una falla de verificacion en la funcion \texttt{parse} mencionada en la bibliografia, lo más grave es que nunca se chequea ninguna de las siguientes cosas:
\begin{itemize}
	\item Unicidad de voto por boleta
	\item Cantidad de votos por candidato sea a lo sumo 1
\end{itemize}

Continuando con la descripción de la vulnerabilidad, en otra sección vital del código, el recuento de votos, se encuentra el siguiente código:

\begin{lstlisting}
class Count(object):

 ...
   def add_selection(self, selection, RFIDserial=None):
        if not RFIDserial or not self.serial_exists(RFIDserial):
            for candidate in selection.candidates:
                self.results[candidate.party_code, candidate.category_code] += 1
            if RFIDserial:
                self._RFIDserials.append(RFIDserial)
  ...
        else:
            raise RepeatedSerial()
\end{lstlisting}

Aquí puede verse una validación sobre la identificación única de cada boleta, pero nunca se realiza la validación pertinente a cada boleta en sí misma acerca de los 2 puntos mencionados anteriormente.

Esta vulnerabilidad, permitía al atacante, escribir sobre una boleta electrónica mediante un smartphone, varios votos a un mismo candidato y esto pasaría desapercibido por ambos módulos del sistema mostrados arriba, constituyendo una fuerte vulnerabilidad.

A continuación se mencionan posibles soluciones a este problema:

\begin{itemize}
	\item Verificar que la escritura de los votos hayan provenido de una máquina real de vot.ar, por ejemplo usando claves privadas
	\item Verificar al momento de la lectura y recuento de la boleta las cuestiones mencionadas
	\item Como método de detección se pueden comparar la cantidad de boletas y la cantidad de votos registrados por máquina
\end{itemize}


\subsubsection{Ataques de denegación de servicio por botnets causando que no se contabilicen votos}

El mecanismo de transmisión de datos utilizado en el sistema Vot.Ar consiste básicamente en una conexión entre cada terminal de votación y un servidor central que recibe los votos.

Si alguien con acceso a la red de alguna máquina de votacion quisiera atacar la disponibilidad del servidor para, por ejemplo, retrasar los resultados, podría obtener la IP o nombre del servidor central y mediante alguna red de botnets obtenida por otro medio, realizar ataques de denegación de servicio y posiblemente explotar vulnerabilidades en el servidor. Por ejemplo podría realizarse un \texttt{SYN-Flood attack}, alterando la disponibilidad del servidor.


\subsubsection{Transmisión de datos y mecanismos de seguridad}
\footnote{\url{http://ivan.barreraoro.com.ar/vot-ar-una-mala-eleccion/}}
\footnote{\url{https://blog.smaldone.com.ar/2016/05/03/el-dia-que-el-sistema-de-voto-electronico-vot-ar-fue-vulnerado/}}
Los mecanismos de transmisión y seguridad propuestos para el sistema Vot.Ar son como siguen:
\begin{itemize}
	\item La terminal de votación es conectada a internet y un técnico de la empresa encargada inicia un software especial de transmisión en ellas. Esta conexión a internet puede variar, desde una conexión cableada del colegio hasta una conexión 3G/4G.
	\item Este software de transmisión se conecta a un servidor de MSA y luego de realizar el recuento físico de boletas apoyando el RFID en la máquina, estos datos se envían al servidor.
\end{itemize}

La propuesta de la empresa para asegurar la transmisión es la utilización de certificados SSL. Cada terminal cuenta con su propio certificado X509 y su clave privada. Esto permite la autenticación de la terminal frente al servidor y es también utilizado para verificar la integridad de los datos.

Claramente es vital que la distribución e instalación de estos certificados en las terminales sea un proceso bien diseñado ya que la información que manejan es muy sensible. Cualquiera que tenga acceso de manera malintencionada a esta información podría hacerse pasar por una terminal y subir datos espurios al servidor.

A continuación se listan algunos de los problemas encontrados en este mecanismo.

\subsubsection{Exceso de confianza en los delegados de instalación de los certificados}

Según las fuentes, los técnicos eran contratados sin ni siquiera una entrevista presencial. Dadas las vulnerabilidades enunciadas a continuación cualquiera de estos técnicos tenía acceso a información que podía comprometer la elección.

\subsubsection{Elecciones defectuosas al momento de distribuir claves privadas}

La distribución de las claves privadas fue encarada de una forma desprolija e insegura.\\

Los certificados estaban accesibles mediante una aplicación web con URLs muy previsibles:\\

\begin{itemize}
	\item \texttt{https://caba.operaciones.com.ar/media/certificados/CABA\_(COMUNA)\_(NUMERO)-(NOMBRE).tar.gz}
\end{itemize}

El acceso a la página era mediante usuario y contraseña de cada técnico, con lo cual, una vez descubierto este patrón, un solo técnico podía descargar todos los certificados.

\subsubsection{Elecciónes defectuosas al momento de asignar contraseñas}
Para empeorar aún mas las cosas, los nombres de usuario y claves de cada técnico eran trivialmente predecibles, ya que consistian en lo siguiente:

\begin{itemize}
	\item Username: (apellido)+(nombre) en minúscula sin caracteres especiales
	\item Password: (emailDelTecnico)
\end{itemize}

Esto estaba especificado en los mismos manuales de capacitación que recibían todos los técnicos. Cualquiera con acceso a alguna planilla de los datos de los técnicos (por ejemplo el personal de RRHH), podrían haber suplantado la identidad de alguno o varios de ellos para cometer ilícitos en su nombre.

\subsubsection{Falta de limitación de poder a un individuo con acceso a los datos}

Estos últimos temas mencionados también deben analizarse en el marco de la confianza que se deposita en una sola persona. Hubiera sido mejor que la responsabilidad hubiera sido compartida entre varias personas, o al menos, las personas responsables deberían ser controladas de algún modo.

\subsubsection{Ataques realizados al sistema de certificados de Vot.Ar}
Dadas las vulnerabilidades mencionadas anteriormente, la lista de certificados fue filtrada días antes del comicio de 2015. Por otro lado se filtraron además los datos personales de los técnicos, haciendo suponer otras vulnerabilidades.

\subsection{Vulnerabilidades en los protocolos de seguridad de las redes}
Lo siguiente fue extraído de un informe de VITA \footnote{(VIRGINIA INFORMATION TECHNOLOGIES AGENCY)} acerca de un sistema de votación utilizado en Virginia, USA \footnote{\url{http://elections.virginia.gov/WebDocs/VotingEquipReport/WINVote-final.pdf}}.\\
Las terminales de votación utilizadas en este sistema, utilizan la red Wifi para comunicarse. Sin embargo, el protocolo de seguridad de la red utilizado es \texttt{WEP}\footnote{Deprecado por la IEEE en 2004}, además, cada terminal \texttt{broadcastea} su nombre de red (SSID) permitiendo un facil escaneo de las terminales. Se realizó un ataque aprovechando este débil esquema de seguridad, que se detalla a continuación.\\
Luego de obtener la lista de nombres de las terminales escaneando las SSID, se procedió a escuchar la red por aproximadamente dos minutos para obtener un \texttt{packet-trace} de la comunicación de red. Utilizando esta información, fue posible explotar una vulnerabilidad del protocolo WEP y crackear la clave, en este caso la clave era \textbf{abcde}. Una vez que se accedió a la red de la terminal, se realizaron análisis de vulnerabilidades con las herramientas NMap\footnote{\url{https://nmap.org/book/man-port-scanning-techniques.html}} y Nessus\footnote{\url{http://www.tenable.com/products/nessus-vulnerability-scanner}}, cuyo resultado mostró que habia varios puertos abiertos y distintas vulnerabilidades. Como se mencionó en la parte de almacenamiento, la combinación de vulnerabilidades en varios frentes de este sistema permitió un reemplazo total de la base de datos de votación de la terminal.

\subsection{Votación por internet: Debilidades en la transmisión de datos}
El envío de boletas para el voto por internet puede pasar por muchos y diferentes servidores, logrando que el sistema no tenga la certeza de que la boleta recibida por el elector, es la boleta emitida para ser usada durante la votación. Además los emails pueden ser interceptados, leídos y modificados en el tránsito violando todos los requerimientos para realizar una votación


% ESTO NO ES SOBRE E-VOTING, ES SOBRE SISTEMAS NORMALES
% Por ejemplo, en Estados Unidos los estándares de transimisión de votos son muy variables. Cada estado elije las máquinas y los métodos para usar y no se rigen por ningún mínimo de seguridad común. Hay un requerimiento estádistico de un error no mayor a 1 en 1 millón, que aparentemente no se cumple\footnote{Douglas W. Jones, \textit{Problems with Voting Systems and the Applicable Standards}, Testimony before the U.S. House of Representatives' Committee on Science, \url{http://homepage.cs.uiowa.edu/~jones/voting/congress.html}}.


\pagebreak
\section{Privacidad del voto}

\pagebreak
\section{Auditabilidad del sistema}

\pagebreak
\section{Comprobantes}
En inglés se habla de un “paper trail”, literalmente “rastro de papel”, cuando una operación realizada tiene comprobantes en papel verifican que se realizó y cuál fue el resultado. En el acto de votación electrónica este rastro de papel es siempre un adicional a los registros guardados por las varias máquinas encargadas de realizar la elección y tiene la ventaja de ser entendible para cualquiera (sin necesidad de entrenamiento en informática), posiblemente duradero en el tiempo, y de funcionar como back up en caso de que algo fallara con el sistema electrónico.

Hay dos momentos del proceso de votación donde el votante necesita algún tipo de comprobación de su acto: por un lado, al emitir el voto es necesario saber que el proceso fue exitoso y finalizó; y por el otro (y esto no sucede cuando se vota con boletas de papel) el votante puede querer comprobar que efectivamente su voto se contó, y se contó para el candidato elegido. Los sistemas de votación ‘End-to-end’ le permiten verificar ambas instancias.

\subsection{Ejemplo de un End-to-end diseñado recientemente}

Desarrollado en Texas para suplir las fallas de las máquinas DRE compradas 10 años antes, y frente a reticencia de las autoridades a volver a boletas con marcas manuales que ya les habían traído problemas de ambigüedad en el pasado.
El sistema consiste en software diseñado para ser usado sobre hardware comercial, y para permitir que el votante pueda verificar su voto (además de algunos “constraints” más dados por la forma de las elecciones en Estados Unidos, como múltiples días de votación y grandes cantidades listas de candidatos)\cite{star}.

El diseño de este sistema incluye back ups en papel en todas las instancias: al llegar, el votante se registra con su documento y recibe un sticker duplicado con un código de barras. Una de las copias queda en un registro en papel con su firma, y la segunda se escanea en la máquina controladora de la red interna del centro de votación. Esa computadora le entrega un PIN de 5 dígitos, único a esa central, que el usuario ingresará en la máquina de votación propiamente dicha. Al terminar de votar recibirá un “ticket” con dos copias de su voto, una en texto plano con un número de secuencia identificatorio, y otra con un hash del voto encriptado y la fecha, hora e identificador del lugar de votación. El votante deposita la que tiene el voto en claro (luego de verificar que sea el correcto) en la urna, y se queda con el otro. Si el voto no es depositado allí, no es válido. Con el hash que se llevó en la segunda copia, el votante puede entrar en el sitio del condado y verificar que se haya contado como válido.

El hash que se lleva el votante es de la siguiente forma: $z_i = H(E_K(v),m,z_i-1)$ con $H$ la función de hash, $E_k$ el algoritmo de encriptación, $m$ el identificador de la terminal de votación y $z_i - 1$ el hash anterior. De esta forma el hash no contiene la información del voto en forma directa.

Estos hashes se conservan en la máquina con un timestamp del momento en que se realizó el voto, y las claves de encriptación se guardan fuera del sistema, combinando claves públicas de un número determinado de oficiales administrativos de la votación.

Este sistema de votación tiene un rastro muy completo del proceso, pero notablemente, no funciona sólo como back up si no que agrega una medida de seguridad extra: no se cuenta ningún voto registrado electrónicamente si el número de serie equivalente no está en la urna física. Esto anula la posibilidad de modificar los votos guardados en la máquina de forma que se multipliquen o pensar en un ataque que borre los votos guardados ya que la urna se abre de todos modos y es la que tiene el dicho final sobre qué votos se cuentan.

De haber alguna discrepancia de votos, es relativamente fácil aislarla a algún voto en particular.

\subsection{Sistemas electrónicos sobre boleta de papel}
Otro tipo de sistema end-to-end en Estados Unidos, es Scantegrity\cite{scantegrity} o Prêt à Voter, que agregan una capa de inteligencia electrónica a un sistema de boletas anotadas a mano.
El objetivo de Scantegrity era reemplazar los DRE en uso y mejorar las boletas en papel leídas y contadas en forma electrónica con el agregado de un rastro de papel.
El sistema ofrece seguridad para el votante de que su elección fue contada proveyendo un comprobante de voto con un número de serie y un código identificatorio de su candidato. Los códigos son asignados en cada boleta impresa al azar y sólo son visibles cuando el votante marca en la boleta con un marcador especial que revela la “tinta invisible”. El votante luego copia ese código a una porción removible de la boleta. Una vez que el voto fue aceptado por las autoridades de mesa, éstos pasan un marcador con otra tinta especial por el sector removible para que quede visible el número de serie de la boleta, y el votante se lleva este comprobante. Al finalizar la elección éste puede chequear en el sitio del condado que el número de serie de su boleta esté asociado al código que anotó como el correspondiente a su candidato. Los resultados de la elección se calculan de forma reproducible a partir de los publicados en internet, pero la lectura de las boletas se sigue haciendo en forma automatizada con los lectores ya existentes.

Los riesgos posibles con respecto a la generación de comprobantes son dos: que el comprobante contenga información respecto del voto, rompiendo la confidencialidad del acto y posibilitando la venta de votos; y la alteración del resultado final de la elección convenciendo al votante de que votó por quién quería cuando en realidad se generó otro voto.
En este paper\cite{kelsey} se describe una forma de reemplazar las boletas de Pret a Voter por unas falsas para que el código ingresado no sea el deseado, si no el de un candidato en particular. Este ataque no es técnicamente informático, si no que depende de los administrativos que entregan las boletas y la forma de “ticket” que elija el votante.


\subsection{Problemas de usabilidad}

Idealmente, es deseable implementar todo aquello que vuelva al sistema más seguro. Pero en la práctica hay otros factores que afectan sobre la decisión de agregar características nuevas al sistema de votación. En particular, hay un estudio hecho sobre los sistemas mencionados que funcionan como agregado a las boletas de papel que, si bien ayudan estrictamente hablando de seguridad, son tan difíciles de usar para los votantes que el porcentaje de votos exitosos puede ser tan bajo como el 58\%\cite{usability}. Esto quiere decir que un porcentaje importante de los votantes no puede elegir a un candidato exitosamente porque no comprendió el uso correcto de alguna porción del sistema. Así mismo, registraron tiempos más largos de uso y niveles de frustración altos entre las personas testeadas. Y agregan que la cantidad de gente que efectivamente realiza la verificación posterior de su voto es muy baja.

\subsection{Paper trails parciales}

Fuera del contexto de los sistemas end-to-end hay otras opciones utilizadas. Hay sistemas que utilizan papel en forma parcial, por ejemplo, combinando una impresora con un DRE, que devuelve un comprobante del voto que le permite al votante verificar que fuera el candidato correcto. En ocasiones, ese comprobante puede ser metido en una urna y usado para contabilizar votos de haber problema con el sistema electrónico, pero no es siempre así.

En el caso de que la impresión sea utilizada como backup al sistema electrónico guardándose para posterior verificación, el sistema se denomina VVPAT (Voter-verified paper audit trail), este es el tipo de sistema a implementarse en india\cite{indianExpress} y en uso en algunos estados de Estados Unidos\cite{nevada}. Este tipo de sistemas no le permiten saber al votante si su voto se contó, pero dan un respaldo en caso de haber habido problemas con el resultado electrónico, incluyendo si fuera necesario hacer el recuento de nuevo.
Se implementa de varias formas distintas: puede ser que el votante tenga que cortar el recibo y depositarlo en una urna, como en STAR, o puede ser que el votante nunca toque el recibo, pero si lo vea a través de un cristal para verificarlo, como se hace en Nevada, EEUU\cite{saltman}.

El caso de la Ciudad de Buenos Aires es un VVPAT con ciertas limitaciones. Por un lado, sirve para verificar que el candidato seleccionado es el mismo que figura en la boleta, gracias a la impresión del voto en texto plano, pero no se utiliza el valor impreso en la boleta, si no el que está en el chip que lee la misma máquina para contar\cite{votar}. De forma que las boletas en papel no están sirviendo como verificación a menos que haya un recuento específico.


\subsection{Lugares sin paper trail}

Las máquinas utilizadas en algunas partes de Estados Unidos\cite{stanfordNews} y en todo Brasil son DRE, que como se viene viendo, no entrega ningún tipo de comprobante por sí sola\cite{aranha}. El hecho de que no imprimen ni usan papel se comunica como ventaja a nivel de ahorro de recursos.

En Nevada se vio como una deficiencia de seguridad no tener ningún backup al conteo electrónico, fue ese el estado del que se habló más arriba, y por eso implementó una combinación de DREs con impresoras para generar un VVPAT. Varios otros estados siguieron su ejemplo, como California uno de los estados con más representantes del país, y se está extendiendo por el territorio.

\subsection{Objeciones al paper trail como se implementa habitualmente}

Si bien la literatura parece estar en acuerdo sobre la necesidad de mantener un rastro de papel de las elecciones, hay discusiones existentes sobre la forma de implementarlo. Por un lado, hay algunos autores que remarcan que las máquinas tienen grandes beneficios de accesibilidad, pero que éstos son contrarrestados agregando un recibo de papel con un formato distinto al de la pantalla, en ocasiones muy largo. Todo esto podría dificultar la verificación por personas con limitaciones de lecto-escritura, quienes tuvieran problemas de movilidad o de visión\cite{saltman}.



% \section{Bibliografía}

% \printbibliography


\end{document}
